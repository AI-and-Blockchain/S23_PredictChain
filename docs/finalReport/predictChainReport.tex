\documentclass{article}
\usepackage[utf8]{inputenc}
\usepackage[margin=1in]{geometry}
\usepackage{subfiles}
\usepackage{float}

\title{Predict Chain}
\author{Matthew Pisano, Connor Patterson, William Hawkins}
\date{Spring 2023}

\begin{document}

\maketitle

\section{Project Description}

PredictChain is a marketplace for predictive AI models. Users are able to upload datasets for training predictive models,
request that train models be trained on any previously uploaded datasets, or submit queries to those trained models.
These various models will be operated by a central node or nodes with computing resources available. A variety of
models will be made available, ranging from cheap, fast, and simple to more expensive, slower, and more powerful.
This will allow for a large variety of predictive abilities for both simple and complex patterns.  All the past predictions
form these models will be stored on the blockchain for public viewing.

\subsection{Problem Solved}
\emph{Clearly state the problem your project solves}

PredictChain helps to solve one of the main issues that involve AI models today: accessibility.  Our project fulfils this
need in two ways.  Oftentimes, individuals or groups poses data that they would wish to be used in predictive analysis.
However, these people may not have access to the compute capacity to train predictive models on this data.  Additionally,
yet other people have neither access to predictive training data, nor do they have access to computational resources.
PredictChain solved both of these problems simultaneously.

When users upload their datasets, they allow model to be trained on those datasets.  Higher quality datasets will produce
higher quality models.  When users submit parameters for training, they allow the model that their parameters produce to
be used publicly.  Both of these users are rewarded for their work when a model is queried, and it produces a correct prediction.
This encourages users to participate in contributing the resources needed for good predictions, while leaving a public
record for other users to view.

\subsection{Background}
\emph{Why is this an important and relevant problem within the context of AI and Blockchain}

\subsection{Use Case/ Motivating User Story}
\emph{Use case or motivating user story detailing how someone would use the system you have built}

\newline\quad\textbf{User Story \#1}
\\\textit{Scenario:} As a data analyst, I want to have access to trends of various stock markets so that I can create predictive models that will inform my investment strategies

\newline\quad\textbf{User Story \#2}
\\\textit{Scenario:} As a stockbroker, I want to compare my dataset to another dataset against the same model so that I can get an idea as to which is better


\newline\quad\textbf{Use Cases}
\newline For the sake of space, we will only go over two simple Use Cases, one that allows the user to add a dataset, and another that checks a price request for a dataset 
\begin{table}[H]
\caption{Add Dataset}
\label{tab:my-table}
\centering
\begin{tabular}{|p{3cm}|p{8cm}|}
\hline
\textbf{Identifier:} & UC1 \\
\hline
\textbf{Description:} & The user logs in to PredictChain and adds a dataset to their account\\
\hline
\textbf{Actor(s):} & Site User \\
\hline
\textbf{Precondition(s):} & PredictChain is set up properly (connected to the client which is connect to Oracle) \\
\hline
\textbf{Event Flow:} & 
\begin{enumerate}
    \item The user logs in properly into their account (or creates a new account)
    \item The user inputs a link to the dataset file 
    \item The user inputs a name corresponding to that dataset
    \item The user enter in the amount of bytes the dataset file is
    \item The user presses "Submit" 
    \item The user then either sees the corresponding Oracle transaction ID or an error occurred
\end{enumerate} \\
\hline
\textbf{Postcondition(s):} & The website displays the information onto the dashboard corresponding to the outcome \\
\hline
\end{tabular}
\end{table}

\begin{table}[H]
\caption{Dataset Price Request}
\label{tab:my-table}
\centering
\begin{tabular}{|p{3cm}|p{8cm}|}
\hline
\textbf{Identifier:} & UC2 \\
\hline
\textbf{Description:} & The user logs in to PredictChain and checks the price for a dataset that contains a certain amount of bytes\\
\hline
\textbf{Actor(s):} & Site User \\
\hline
\textbf{Precondition(s):} & PredictChain is set up properly (connected to the client which is connect to Oracle) \\
\hline
\textbf{Event Flow:} & 
\begin{enumerate}
    \item The user logs in properly into their account (or creates a new account)
    \item The user inputs the corresponding amount of bytes that a sample file would have to check the price for
    \item The user is then shown the price that it would cost for a user to upload a file that size in bytes
\end{enumerate} \\
\hline
\textbf{Postcondition(s):} & The website displays the information onto the dashboard corresponding to the input \\
\hline
\end{tabular}
\end{table}

\section{Implementation Details}

\emph{Provide details of how you implemented your solution. Please add as many diagrams as necessary and explain the
diagrams you added in the text.
Any external libraries used and rationale for using them
Links to all the resources produced (GitHub source code, demo link, link to the recorded video of the demo, etc.)}

The structure of PredictChain is primarily broken up into two parts: the client and the oracle.  Both of these parts interact
with each other through the blockchain.

\subsection{The UI}
Although not directly related to the AI nor the blockchain parts of the product, it is still important to discuss the UI aspect of our project. The UI creates the face of PredictChain by having multiple pages that create a more pleasant experience for the user. For example, the home page talks about our mission, how we differ from our competitors and example model sets we provide. The addition of FAQ pages and Meet the Team pages allow users to understand who created PredictChain, as well as get answers about any privacy/security concerns they might have. Lastly, users can create an account or login to a pre-existing account. Then users are able to use PredictChain to its fullest functionality such as adding datasets, checking prices, or querying models with provided datasets. The UI talks to the client in order to get this data and then provides it to the user in a nice manner.

\subsection{The Client}

The client serves as a middleman between the front end user interface and the
blockchain.  It is run as a server, serving UI content to the user, taking in requests from the UI, and parsing those
requests into a form suitable for both the blockchain and for the oracle.  Additionally, the client constantly polls
for updates coming from the oracle, through the blockchain.  These updates are queued and sent to the front end upon request.
This allows the user to both interact with the blockchain and to see important updates that come from it.

\subsection{The Oracle}

The oracle accomplishes the majority of the other tasks that this project requires.  It constantly polls for updates
coming from the client, through the blockchain.  Upon receiving these updates, it begins the execution of one of its
main operations.  These are:

\begin{itemize}
    \item Downloading a user-specified dataset and saving it
    \item Training one of the raw models based on user inputted parameters
    \item Querying one of the trained models on user inputted data and comparing it to the real-world result
\end{itemize}

After each of these operations, the oracle sends out several transactions.  These can be either rewards to contributors
of a model or confirmations/results of the operation that has been performed.

When working with user-submitted datasets, the oracle uses a handler to manage the operations performed on that dataset.
The handler can save datasets to a specified environment, load datasets from a specified environment, parse that dataset
as a pandas dataframe, and split the dataset by the values of one of its attributes.  The environments that the handler's
recognize are \textit{local} and \textit{ipfs}.  When using either of these environments, the handler abstracts away the
complexities of working with either of them into a unified interface.

When working with user-trained models, the oracle uses a similar, common interface.  This interface can create the model
architecture, train the model on a selected dataset, query the trained model, evaluate its performance, save the model,
and load it back from a specified environment.  When creating and training a model, the interface chooses among a group of
archetype or template models.  These models can be a:

\begin{itemize}
    \item Multi-layered perceptron neural network
    \item Recurrent neural network
    \item Long short-term memory neural networks
    \item Gated recurrent unit neural networks
\end{itemize}

Each of these models has a \textit{model\_complexity} attribute.  This is a simple float value, designed to give users
a general idea of how performant a model can be once trained and serves as a method of calculating the cost of using
or training that given model.  The attribute itself is calculated using the size of the network and a linter multiplier to
account for more complex model architectures.  For models like GRUs or LSTMs, the complexity is higher as they are mode complex, and often
better performing models.  For models like MLPs, the complexity is lower.  This gives the desired effect of faster, simpler
models being cheaper than the slower, more complex models without any heavy calculations.   The interface abstracts most
of the complexities of training, querying, and evaluating these models.  The only difference between them is the inclusion
of several optional parameters.

\subsection{The Blockchain}

In PredictChain, the blockchain serves as both a records keeper and a messenger between the client and the oracle.
This is accomplished by using transactions as a form of direct communication.  With every transaction sent is a note.
This note is a json-encoded string (encoded in base64) that communicate information about the operation that the transaction
is requesting be performed and arguments relevant to that operation.  The operations in the note are represented by a series
of op codes.  These codes are abbreviations of the operation name enclosed in angle brackets, for example \textit{<QUERY\_MODEL>}.
The arguments to these operations are represented as a named dictionary, with each key being the name of the argument and each
value being the argument itself.  This named strategy allows the program to be very flexible without worrying about the exact
ordering of the arguments.  Blockchain is quite useful in its role due to its immutability and its transparency.  Using
a blockchain means that all requests are permanently stored and public, so other users can see what type of models are useful
for specific datasets and what results those models have produced.

\section{Evaluation}

\emph{How do you know your solution works? What tests have you performed, and what results have you obtained?
If you have acquired any users to try out your system, summarize their reactions and feedback}

\section{Conclusion}

\subsection{Summary}
\emph{A short summary of the project}

\subsection{Limitations}
\emph{Limitations of the work}

\subsection{Future Work}
\emph{Potential future work}

\cite{test}
\bibliographystyle{abbrv}
\bibliography{references}

\end{document}