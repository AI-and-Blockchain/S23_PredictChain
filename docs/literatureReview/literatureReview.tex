\documentclass{article}
\usepackage{hyperref}
\usepackage{float}
\usepackage[utf8]{inputenc}
\usepackage[margin=1in]{geometry}
\usepackage{makecell}
\usepackage{booktabs}
\newcommand{\tabitem}{\\~~\llap{\textbullet}~~}
\title{Predict Chain}
\author{Matthew Pisano, William Hawkins}
\date{Spring 2023}

\begin{document}

    \maketitle

    \section{Introduction}

    Throughout our project, we have built upon, or have related to, the ideas of many other projects and papers.
    Some of these relations were explicit, such as our work with, and utilization of, various types of neural
    network.  Others were implicit, like the relation that our project has to the many other implementations of
    blockchain-based predictive projects.  It is important to acknowledge the influences that other works have
    had on our project, and it is equally important to analyze the similarities between our project and the
    other papers that we have referenced here.  Through this, we hope to better understand the perspective of
    others on similar topics to our project and hopefully improve upon our work and techniques in the future.


    \section{Related Work}
    % Please feel free to add any subsections as necessary to elaborate on different key facets of the related work.
    % However, your writing must follow a cohesive structure (i.e., it should not read like an annotated bibliography!).
    % But please include a table to compare specific attributes of the related work with your project.
    % Use the references.bib file to include the citations in biblatex format.

    \subsection{Sharing Updatable Models}

    Decentralized \& Collaborative AI on Blockchain~\cite{sharingModels} is a great example of a project that combines
    both artificial intelligence and blockchain technologies in a similar manner to our own.  In this paper, Microsoft
    outlines a blockchain-based predictive marketplace that utilizes AI for predictive analysis.  In this system, there
    exists an initial model that is trained on the data collaboratively provided by multiple users.  To encourage
    the submission of high-quality data, they propose an incentive mechanism that rewards honest users, and punishes
    dishonest ones.  This model is trained off-chain on the client's machine.  All the logic involving the
    data submission and incentives are done within an Ethereum smart contract.

    \paragraph{The Model}
    This system leaves the exact definition of the model flexible between different clients running the system.  They
    suggest that the specific class of model would be some type of neural network, giving examples of CRF-RNNs and
    Support Vector Machines.  Notably, both of these models are geared towards regression and series analysis as the
    focus of this system is on predictive analysis.  Depending on the model that is used, the authors mention that the
    corresponding mechanism would change from model to model.

    \paragraph{Incentives}
    Similarly to the models, the authors leave the incentive mechanism of the project also without an exact definition.
    This is to give end users flexibility in their individual implementations.  They do give several examples and
    specifications of this mechanism, however.  These examples fall into two categories: gamification and monetary
    incentives.  In the gamification strategy, they suggest that there should be no monetary incentive for contributing
    data.  This would create a system that encourages a collaborative effort in building a model for the common good.
    In this strategy, instead of money, they would have tokens or badges that are awarded to users for good data.  The
    contract would evaluate contributed data as 'good' if it increases the accuracy of the model on a testing set.
    For the monetary route, they suggest that users who choose to upload data may have to also submit some stake into
    the contract.  Using this stake as collateral, the contract uses a similar evaluation of 'goodness' to distribute
    awards or punishments.

    \paragraph{Relation to Our Work}
    This paper shares a very large similarity with our project.  However, while the core ideas are quite similar, there
    are some notable differences.  Both of the projects involve creating a model, open to public query, and trained on
    user-provided data.  Both of these projects have the goal of making machine learning predictions more accessible
    to more people.  In comparison, our project is more centralized and with a larger focus on incentives.  For
    example, in our project, models are trained on a special, central node instead of on a node managed by a client.
    Each model in our project has only one user-uploaded dataset as its source instead of each model being a collaborative
    effort from many users.  In PredictChain, there is a greater focus on costs and incentives than in the sharing models
    project.  Specifically, our project is built around encouraging usage through the possibility of monetary reward.
    But, in the author's system, the usage is achieved by either gamification or a monetary incentive.  Additionally,
    queries are not free in PredictChain as they are in Decentralized \& Collaborative AI on Blockchain.  In both systems,
    individual users make contributions in the form of data, one initial user controls the model hyperparameters and
    which dataset the model will be trained on, and all users can query the publicly available models.

    \subsection{The Price You Pay For Trust}

    Blockchain Enabled AI Marketplace: The Price You Pay For Trust~\cite{priceOfTrust} is another example of a marketplace
    that uses both AI and blockchain technologies effectively.  One of the main focuses of this project is on privacy.
    It explicitly gives examples of health or insurance entities needing models trained on confidential data.  The principles
    that this project is built around are \textit{trust}, \textit{fairness}, and \textit{auditability}.

    \paragraph{Datasets}
    The datasets in this system are unique since the bulk of the data is not tied directly to any one training request.
    In this project, a user uploads only a small chunk of data.  This data will then be used as the validation set for
    the training.  The majority of the training data for this query is pulled from private sources.  This serves the purpose
    of only requiring users to upload a small portion of their data.  However, this has the disadvantage of requiring
    external data to train models.  To acomodate this, the authors implement several layers of security to protect private
    data sources.

    \paragraph{Models}
    The models that this paper targets are much less integrated than those of other papers.  Also unlike other implementations,
    these models are trained in a decentralized nature.  This training strategy is reflective of federated learning.  No
    one model is given all the data in bulk, it is only given sections with the confidential information stripped out.
    To add to their security, they also use hashing algorithms to further secure their data.  Additionally, the models
    themselves are eventually shared with the requesting user along with the model results.

    \paragraph{Blockchain Communication}
    Like other projects, this implementation uses a significant off-chain component.  This component communicates
    with the private, permissioned blockchain that this system uses.  This is to allow for more complexities
    in the training of the models as on-chain training has significant speed and memory constraints.  Additionally,
    the authors utilize Hyperledger Fabric to aid in the distribution of the data between the nodes and the aggregator.
    By using this bridge, data can be selectively distributed to only the nodes that need the data for training will
    have access to it.

    \paragraph{Relation to Our Work}
    This paper also has a notable similarity to PredictChain.  Both of these projects utilize blockchain as an important
    means of communication and record-keeping in an AI marketplace environment.  Similar to the updatable models
    project, this project implements this idea in a more decentralized manner.  Instead of only keeping models on the
    client nodes, like PredictChain, this project allows users to query the content of the models directly, provided that
    the users is the one who asked for the training or if the user is a training node.  Also unlike our project,
    the training of these models is a collaborative effort.  Instead of one, monolithic node, this project used
    a federated style of learning.  Finally, the authors have implemented a data system that is more decentralized than
    monolithic.  Instead of users uploading data directly, it is pulled form a variety of sources, including the original
    user.  In this system, the private data owners must submit bids to the aggregator for the chance that their data
    gets included in a trained model.

    \subsection{Blockchain Prediction Markets for Renewable Energy}

    Towards the Use of Blockchain Prediction Markets on Forecasting Wind Power~\cite{windForcasting} is a paper that lays the
    groundwork for a use-case of blockchain empowered prediction markets. Like blockchain technology, renewable energy has
    recently ballooned in use. Many people are drawn to both of these innovations for some of the same reasons:
    they desire a new way of utilizing technology that breaks from the old system and provides them with some benefit.
    Like renewables replacing fossil fuels, many see blockchain replacing centralization as a necessary step to a better,
    freer, more sustainable future.  This paper combines these two fields in a way that utilizes their combined utility
    and benefits.

    \paragraph{Motives}
    A key problem with many forms of renewable energy sources are their unpredictable supply. Wind turbines need wind,
    solar panels need sunshine. Being able to better forecast the output of these energy production methods at a various
    time or place can help alleviate the pitfalls that prevent them from growing in market share.  At the time of publication
    the authors, Shamsi and Cuffe, note that the use of blockchain based prediction markets in renewable energy forecasting
    had not yet been explored.

    \paragraph{Implementation}
    On their platform, anyone can propose an event and request predictions for that event with the promise of providing
    a reward.  Reporters then propose several predictions over the course of the seven-day window that the market is
    active.  In this implementation, the hosts of this predictive market do not take responsibility for the training
    and execution of the predictive models.  This is all done on the side of the reporters.  By using their framework,
    reporters who provide correct data are rewarded and those who provide inconsistent data are punished.

    \paragraph{The Oracle Problem}
    Shamsi and Cuffe discuss various frameworks that they can use to organize this system. One crowdsourcing proposal, they
    say, lacks proper incentives. While another prediction market with acceptable incentives is too centralized. The authors
    then discuss the \textit{Augur} Platform, an Ethereum based prediction market. The Augur system seeks to deal with a
    key issue with decentralized prediction markets: The oracle problem. This is the problem of getting new outside data
    into the platform. Augur gives a basis for "Independent participants [to] report, dispute and validate the data
    through a decentralized set of smart contracts".  For their paper, the authors have based their framework off
    of Augur's.  They hope to expand upon its proven reliability with their own improvements.

    \paragraph{Relation To Our Work}
    Both PredictChain and this paper enable users to gain access to better forecasting of events. In PredictChain, users
    are able to query pre-existing models as well as contribute their own data to train these models. This collaboration
    improves performance and accessibility to predictive capabilities. In the predictive marketplace for renewable energy,
    users can observe the spread prices on trades and calculate what the market clearing probabilities are for any such
    event. The more contracts that are sold and bought on the marketplace, the closer the spread price will be to the true
    event probability or value.  In other ways, this paper differs from our project. In PredictChain, a central oracle
    gathers new data to determine the accuracy of model queries. More specifically, in stock market prediction for example, a
    user may specify a model to query. That user can also provide the data to call the model on e.g. stock data for
    \$AAPL from 4/17/2023-4/19/2023 if we assume this recurrent model is using a three-day window. The oracle would then
    gather the true stock market price for the target date, 4/20/2023, and record the model's accuracy.  This is in contrast
    to the decentralized nature of the Augur oracle that the authors use.  While our oracle is trusted, the nodes of
    the Augur oracle are untrusted, with different oracles sometimes reporting conflicting events and disputing each other.
    This choice represents a tradeoff between reliability with simplicity.


    \subsection{Stock Predictions}

    This next paper is closely aligned with the example dataset that we are using for PredictChain.  They focus on
    using big data to train predictive machine learning models to forecast the short-term future behavior of stocks
    within the Chinese stock market~\cite{deepPrediction}.  Within their paper, they utilize several techniques; designed
    to cut down on the noise that is inherent to the stock market to allow for better predictions.  In their paper,
    they focus mainly on three methods that they use to accomplish this goal:

    \begin{itemize}
        \item Creating a new dataset, gathered primarily from the \textit{Tushare} API
        \item Performing feature engineering on their dataset to ensure that it can be better used by their model
        \item Customizing a Long Short-Term Memory model for the actual predictions
    \end{itemize}

    For their dataset, they had gathered the past performance of 3558 stocks from the Chinese stock market over
    a period of two years.  Each entry in their dataset contains many attributes including the daily price of a
    given stock and the ID of that stock.  In order to gather this data, they utilized the \textit{Tushare} API, along
    with web-scraping from \textit{Sina Finance} and the \textit{SWS Research} website.

    \paragraph{Feature Engineering}
    As part of their feature engineering, they utilized three primary techniques.  First, they applied feature extension
    to their dataset.  Through the use of this technique, they added additional meta-attributes to the entries, such
    as polarizing or calculating the fluctuation percentage.  Adding this meta-data to the dataset helps to give the model
    a more complex picture of the stock and how it performs over time.  Next, they eliminated some features by using the
    Recursive Feature Elimination (RFE) algorithm.  This algorithm evaluated each feature based on the degree to which
    that feature influenced the stock's performance.  By eliminating unnecessary features, they allow their model to
    pay more attention to the most influential features, improving its predictive capabilities.

    \paragraph{Predictive Models}
    Finally, they developed their models.  They used a two layered LSTM model, with the model only having an input layer
    and an output layer.  As for their output, they kept the output as simple as possible to make sure the model
    was focussed only on predicting the movement of the stock.  The model either outputted a one for the stock going up at
    a given time step, or a zero if the model believed the stock would go down.  Their hope is that this simplicity
    will also simplify the complexity of the information that investors will have to interact with.

    \paragraph{Relation to Our Work}
    As mentioned previously, the general purpose of this work mirrors part of the functionality of our project.
    The main difference here, being that it is much more focused on the predictive aspect.  Along with their
    LSTM model, they test several other models on their dataset as a comparison.  Out of these models,
    the LSTM performed the best, with the multi-layered perceptron and the other, non-neural network, algorithms
    falling behind.  Their observed performance matches our initial assumption and the modifications we had made
    after reviewing~\cite{recurrentModeling}. One notable feature that this paper has that our project has not fully implemented
    is their complex preprocessing of the dataset.  At present, PredictChain leaves the majority of dataset preprocessing
    up to the uploading user.  That is, our project does not perform any feature engineering before submitting the dataset
    to models.  In place of this, we allow the user to split the dataset about the value of any given attribute.  The usage
    of our $sub\_split\_attrib$ allows users to partially preprocess their data dynamically, similar to, but less complete
    than, this paper's implementation.  A future improvement that we could make to our project is to give users the option
    of having feature extension or elimination automatically applied to their dataset after submission.  This change may
    result in yet better performance of our models after training on user-submitted datasets.

    \subsection{Table Comparison}

    As a summary, this chart illustrates the commonalities and differences between our project and these related works:

    \begin{table}[H]
        \begin{center}
            \caption{Comparison of Each Paper}
            \label{tab:relatedSummary}
            \bgroup
            \def\arraystretch{3}
            \begin{tabular}{c|c|c|c}
                \textbf{Paper} & \makecell{\textbf{PredictChain's}\\\textbf{Specification}} & \textbf{Paper's Specification} & \textbf{Commonalities}\\
                \hline
                \makecell{Harris et al.\\2019~\cite{sharingModels}} &
                \makecell[l]{
                    \tabitem Model run and trained\\on special node\\
                    \tabitem Datasests from one\\user only\\
                    \tabitem Always has incentives\\
                    \tabitem Paid queries\\
                } & \makecell[l]{
                    \tabitem Model run and trained\\on client node\\
                    \tabitem Datasets collaborative\\from many users\\
                    \tabitem Has optional\\gamification\\
                    \tabitem Free queries\\
                } & \makecell[l]{
                    \tabitem Community dataset\\contributions\\
                    \tabitem User-specified models\\and hyperparams\\
                    \tabitem All users can query\\public models\\
                }\\\hline
                \makecell[l]{Sarpatwar et al.\\2019~\cite{priceOfTrust}} &
                \makecell[l]{
                    \tabitem Model run and trained\\on special node\\
                    \tabitem Models kept only on\\one node\\
                    \tabitem Datasests from one\\user only\\
                } & \makecell[l]{
                    \tabitem Models run and trained\\on multiple nodes\\in a federated manner\\
                    \tabitem Models can be shared\\between nodes for training\\
                    \tabitem Datasets are submitted\\by users with additional\\data submitted by\\private soucres\\
                } & \makecell[l]{
                    \tabitem Users can submit\\custom datasets\\
                    \tabitem Data from different\\users can be used for a\\common purpose\\
                }\\\hline
                \makecell[l]{Shamsi et al.\\2020~\cite{windForcasting}} &
                \makecell[l]{
                    \tabitem Central Node reports \\ new data to the oracle \\
                    \tabitem Predictions from models \\ hosted on central node \\
                    \tabitem Models are trained and\\managed by oracle\\
                    \tabitem Flexible scope,\\generalizable to any domain\\
                } &
                \makecell[l]{
                    \tabitem Users report new \\ data to the oracle \\ for contract-settling.  \\
                    \tabitem Users manually submit \\ their predictions via \\ buying contracts \\
                    \tabitem Models are trained and\\managed by reporters\\
                    \tabitem Scope constrained to\\predictions of renewable\\output
                } &
                \makecell[l]{
                    \tabitem Give users greater\\access to predictive\\capabilities \\
                    \tabitem Users submit queries\\for predicting\\future events\\
                } \\\hline
                \makecell[l]{Shen et al.\\2020\cite{deepPrediction}} &
                \makecell[l]{
                    \tabitem Incorporation of\\blockchain to distribute\\predictions to users\\
                    \tabitem Has limited options\\for dynamic preprocessing\\of data\\
                } &
                \makecell[l]{
                    \tabitem Focus prmarily on\\predictions rather than\\accessibility\\
                    \tabitem Has many options for\\preprocessing data before\\ being sent to models\\
                } &
                \makecell[l]{
                    \tabitem Utilize various predicive\\models with different\\advantages and\\disadvantages\\
                    \tabitem Focus on stock market\\predictive data in sample\\datasets\\
                }\\\hline
            \end{tabular}
            \egroup
        \end{center}
    \end{table}


    \section{Background Knowledge}

    \subsection{Long Short-Term Memory Networks}

    Long Short-Term Memory Networks\cite{LSTM} are some of the most influential, and useful, types of network that we work
    with in our project.  These networks introduce several important improvements on classical backpropagation-through-time
    recurrent neural networks.  One of the most notable contributions of LSTMs is their resolution of the 'exploding
    and vanishing gradients' problem.  In RNNs, it is common, and often disruptive, to observe error signals either
    grow to extreme values or disappear entirely when updating the weights of the model.  This either leads to
    extreme unpredictability in the next value that weights will take on, or it leads to the weights not being
    updated at all. In both cases, the effectiveness of the model is greatly impacted.  To help remedy this issue,
    the authors propose an architecture that enforces a 'constant \ldots error flow through internal states' inside the
    model.

    Before this paper, it was well known that traditional RNNs performed very poorly when given long time lagged
    datasets.   In this data, the time window that the input data is set in is several time steps ahead of the window
    that contains the target prediction.  This was so pronounced that, in another paper\cite{weightGuessing}, the authors
    showed that other contemporary RNN algorithms can be outperformed by random guessing.  To show that their new
    architecture is superior in areas that RNNs fall short in, the authors devise a series of six experiments.
    Through these experiments, they show that LSTMs can outperform RNNs on long, minimal time lag data.  Through their
    papers, the authors show that their new LSTMs can perform well in both short and long time lagged situations.

    \paragraph{Relation to Our Work}

    The work of Hochreiter and Schmidhuber has been incredibly useful in our development of PredictChain.  As part of
    our goal, our project is designed to accept a wide variety of datasets and is also designed to train models on
    a wide range of parameters.  Part of this promise of variety is our models' abilities to work with both short
    and long time lagged datasets.  While some of our models fall short in some areas, such as RNNs in a long lagged
    context, others are able to excell in those areas, such as our LSTM and GRU type models.  Through our incorporation
    of the authors' work, we are able to provide a large range of models for a wide variety of use cases.  This helps
    the project to achieve its goal of greater accessibility in the field of predictive modeling.


    \subsection{Recurrent Sequence Modeling}

    One of the core aspects of our project is our usage of various types of neural network for predictive modeling.
    Most of our neural networks have some recurrent capability, specifically our RNN, LSTM, and GRU model types.
    Additionally, the data that we suggest that these models be trained on is sequential data, exactly what these
    types of model were designed for.  One paper that aligns especially closely with this component of our project
    is \textit{Empirical Evaluation of Gated Recurrent Neural Networks on Sequence Modeling}\cite{recurrentModeling}.

    Within this paper, the authors aim to evaluate the relative performance of RNNs, LSTM networks, and GRU networks.
    Similar to our use case, they perform this evaluation using sequential signal data, similar in nature to the
    stock market dataset that we use in our example.  Primarily, they focus on the LSTM and GRU variants of
    recurrent neural networks, rather than vanilla RNNs themselves.  As part of their evaluation, they compare
    the performance of LSTM unit, GRU unit, and tanh unit RNNs on modeling sequences of polyphonic music and
    speech signals.

    As part of their evaluation, they utilize three publicly available music datasets along with two internal speech
    datasets from Ubisoft.  The design of their experiment is to compare the negative log probabilities of the models
    after they are given 20 consecutive samples of audio data and are then tasked to predict the next 10 consecutive
    samples.  In the results that they gathered, they concluded that, overall, the GRU unit outperformed the other
    two evaluated models.  These results are amplified in the Ubisoft datasets, where GRU units significantly
    outperform the others in one of the test cases, and have similar performance to LSTMs in the other.  Additionally,
    in the Ubisoft Dataset B, the negative log-likelihood of the GRU model continues to shrink in a super-linear manner,
    whereas the other models become asymptotic.

    \paragraph{Relation to Our Work}

    This paper both confirmed some of the initial assumptions that we had about the models and changed some others.
    In the design of the project, we had initially marked GRU models to be better than all the other models.
    We denoted this perceived difference in output quality through the constant multiplier we applied to each of their
    $model\_complexity$ values.  After our analysis of this paper, we ran several tests to see if this paper's results
    also applied to our project.  In our tests, we noticed something very similar to the paper.  GRUs and LSTMs
    have more similar performances than we initially thought.  Meanwhile, vanilla RNNs lagged behind.  Based off of
    this evidence, we decided to update these values to better reflect the experimental performance of these models.

    \begin{table}[H]
        \begin{center}
            \caption{$model\_complexity$ Values of Our Models}
            \label{tab:modelComplexity}
            \begin{tabular}{c|c|c}
                \textbf{Model} & \textbf{Initial Value} & \textbf{Final value}\\
                \hline
                MLP & 1 & 1\\
                RNN & 1.5 & 1.7\\
                LSTM & 2 & 2.2\\
                GRU & 2.5 & 2.4
            \end{tabular}
        \end{center}
    \end{table}

    In our changes, we amplified the values related to all RNN-based networks.  We also brought the complexities
    of the LSTM and GRU networks closer together.  This change was made to reflect their relative performances to
    each other and to more traditional RNNs.  Overall, this paper has been important for our project as it has
    helped us to reflect upon our models and how to best evaluate them.

    \subsection{Table Comparison}

    As a summary, this chart illustrates the impact that each of the papers had on our work and our knowledge:

    \begin{table}[H]
        \begin{center}
            \caption{Impact of Each Paper}
            \label{tab:backgroundSummary}
            \bgroup
            \def\arraystretch{3}
            \begin{tabular}{c|c}
                \textbf{Paper} & \textbf{Notable Impact}\\
                \hline
                \makecell{Hochreiter et al. 1997\cite{LSTM}} & \makecell{Encouraged us to expand our support for forcing time lag\\
                    in datasets and expanding the degree to which users\\could customize the lookback windows}\\
                \hline
                \makecell{Chung et al. 2014\cite{recurrentModeling}} &
                    \makecell{Helping us to better understand the relative performances\\of RNNs, LSTMs, and GRU networks}\\
            \end{tabular}
            \egroup
        \end{center}
    \end{table}

    \section{Closing Thoughts}

    After researching and reading these relevant papers, we were able to gain valuable insight into projects similar to
    our own, implement relevant features, and take note of some features that could be added in future revisions of
    PredictChain.  Specifically, we were able to use some of these papers as background knowledge to improve upon
    specific details of our implementation.  Through other, contemporary papers, we were able to use to investigate similar
    projects and inspire possible future work on ours.

    \paragraph{Related Papers}
    Through our review, we have examined several papers that demonstrate similar ideas and architectures to our project
    with slightly different implementations.~\cite{sharingModels} had many similarities to our project, implemented using
    a more decentralized architecture.~\cite{priceOfTrust} takes some of the decentralization ideas of~\cite{sharingModels}
    and expands upon it by structuring their project with security as another main focus.  ~\cite{windForcasting}
    demonstrates a market, similar to ours, but using a more decentralized architecture with a more specific scope.
    Finally, ~\cite{deepPrediction} helped us to better understand some of the future improvements that we could make
    to both our dataset preprocessing and model training.

    \paragraph{Background Papers}
    The papers that we have used for background knowledge helped us to re-revaluate how we rate our models and how we can best
    improve the data processing and training aspects of the project.~\cite{LSTM}~and~\cite{recurrentModeling} helped
    us to better understand how we should rank the potential effectiveness of our models through our usage of
    their $model\_complexity$ values.  Specifically,~\cite{LSTM} encouraged us to expand the degree to
    which users could customize the time lag that is present in the dataset upon model training.

    \paragraph{Future Work}
    After both reading these papers and our own evaluations, we have identified some possible future work that may
    benefit PredictChain.  One major change that we could make to the project is the overhaul of the architecture of
    the project.  This would likely come in the form of some work that makes our project more decentralized in nature.
    This could come in the form of utilizing multiple reporters that manage their own models like~\cite{windForcasting}
    or by using a more decentralized form of dataset sourcing like~\cite{priceOfTrust}.  Adding these would open up more
    opportunities for community contribution.  Another potential improvement that could be introduced with future work is
    offloading more of our logic into smart contracts.  This idea was inspired by the contemporary papers as a whole, as
    they all utilize smart contracts in some way.  This would make more of our logic transparent to users without them having
    to look at our open-source code themselves.  We initially did not choose to do this due to some difficulties in using TEAL,
    but in some future work, incorporating these into our design would make our project more open and decentralized.
    Adding these improvements may make PredictChain a more fleshed-out version of itself without changing the core
    principles of our project.

    One way that smart contracts could be integrated into PredictChain is as a medium for holding competitions between different models hosted on PredictChain. A user could create a Competition and select some number of events to be part of that competition. Other users could then enter models in the Competition. These models would then compete for best performance over the selected events. Once the models and events are selected everything and all predictions can be run through PredictChain with payouts being settled by the smart contract. \cite{brierScore} discusses the Brier Score as a way of comparing binary probabilistic models. In their case discussed in the context of weather events. This provides an added functionality over traditional prediction markets. Models are not only judged on their predictions, but on the confidence of those predictions as well. This also gives users the ease of not having to manually enter predictions for each event allowing models to compete over many events over any period of time. 

    \pagebreak
    \bibliographystyle{abbrv}
    \bibliography{references}

\end{document}