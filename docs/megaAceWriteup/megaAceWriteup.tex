\documentclass{article}
\usepackage[utf8]{inputenc}
\usepackage[margin=1in]{geometry}
\usepackage{subfiles}
\usepackage{float}

\title{Predict Chain}
\author{Matthew Pisano, Connor Patterson, William Hawkins}
\date{Spring 2023}

\begin{document}

    \maketitle

    \section{Problem Summary}

    One of the main issues with artificial intelligence training and development today is accessibility. Oftentimes,
    individuals or groups possess data that they would wish to be used in predictive analysis. However, these people
    may not have access to the compute capacity to train predictive models on this data.  Additionally,yet other
    people have neither access to predictive training data, nor do they have access to adequate computational resources.
    Compounding this issue is the fact that the entities that \textit{do} have these computational resources often keep the results of
    trained models hidden from the public.  Due to this lack of accessibility, similar models are often trained on similar datasets,
    obtaining similar results.  This results in both a notable inefficiency, and the withholding of useful predictions from
    the majority of people.

    \section{Proposed Solution}

    \subsection{Description}

    To help to aid with this issue, we propose PredictChain, a blockchain-based marketplace for predictive AI models.
    Through PredictChain, users are able to upload datasets for training predictive models, request that train models
    be trained on any previously uploaded datasets, or submit queries to those trained models.
    These various models will be operated by a central node or nodes with computing resources available. A variety of
    models will be made available, ranging from cheap, fast, and simple to more expensive, slower, and more powerful.
    This will allow for a large variety of predictive abilities for both simple and complex patterns.  All the past predictions
    form these models will be stored on the blockchain for public viewing.

    \subsection{Implementation}

    The core of PredictChain is implemented using Python and its Algorand SDK.  We have also developed a front end for the
    project in React.  PredictChain is implemented using two primary parts: the client and the oracle.  The client interacts
    with the end user through a front end website.  It takes in user requests and reports back the results of those requests.
    The oracle handles the majority of the backend functionality.  It handles the storing of uploaded datasets, the
    training of models on those datasets, the querying of those models, and the management of payments and rewards to the users.

    In order to facilitate communication between the client and oracle, we use transactions on the Algorand blockchain.
    This is done through JSON-encoded nodes that are attached to the transactions.  The advantage to using transactions as
    our primary mode of internode communication is that it creates an immutable, public record of all requests that
    have been made and all responses that have been returned.  This ensures that, even though users may not have direct
    access to sensitive datasets they can still use the predictions that are produced from that dataset.

    \subsection{Addressing the Problem}

    Our project fulfils the need for a more accessible ML environment in two ways.  It encourages users to generate high
    quality content through the usage of incentives and it ensures that the results of this effort are also publicly
    viewable.  Using this system, people will be able to have access to high quality models to make predictions based
    off of their datasets and people are able to view these predictions, cutting down on wasteful, duplicate work.

    When users upload their datasets, they allow model to be trained on those datasets.  Higher quality datasets will
    produce higher quality models. When users submit parameters for training, they allow the model that their parameters
    produce to be used publicly.  Both of these users are rewarded for their work when a model is queried, and it produces
    a correct prediction.  This encourages users to participate in contributing the resources needed for good predictions,
    while leaving a public record for other users to view through the usage of Algorand transactions.

    \section{Potential Impact}

    Given enough users, this project would have a notable impact on the state of machine learning usage.  Currently,
    the problem of accessibility prevents many people from utilizing machine learning themselves, often leaving them to
    turn to highly private and centralized tech giants.  This project would serve to open up this black-box of
    an industry and encourage the sharing of datasets and parameter configurations to create better models that are open
    to the public for usage.

\end{document}